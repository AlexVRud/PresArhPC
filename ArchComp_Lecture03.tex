\documentclass[pdf,9pt,aspectratio=169]{beamer}
\usepackage[utf8]{inputenc}

\usepackage{DejaVuSans}
\usepackage{DejaVuSerif}
\usepackage{DejaVuSansMono}
\usepackage[T2A]{fontenc}

\usepackage[russian]{babel}

\usepackage{hyperref}
\hypersetup{unicode=true,colorlinks=true}


\usetheme{Madrid}
\usefonttheme[stillsansserifsmall]{serif}
%\usefonttheme[onlylarge]{structurebold}
\usefonttheme[onlylarge]{structureitalicserif}

\title[Архитектура компьютера: Условный оператор]{Архитектура компьютера}
\subtitle{Условный оператор}
\author[А.В. Рудалёв]{Александр Васильевич Рудалёв}
\institute[ВШИТАС САФУ]{ВШИТАС САФУ}
\date[г. Архангельск, 2017 г.]{г. Архангельск, 2017 г.}

\usepackage{wrapfig}

\usepackage{NarfuLectures}

\usepackage{minted}
\usemintedstyle{default} 
\newcommand{\miln}[1]{\mintinline{nasm}{#1}}
\newcommand{\milt}[1]{\mintinline{text}{#1}}
\newcommand{\cl}[1]{\texttt{#1}}

\usepackage{array}
\usepackage{makecell}
\usepackage{tabularx}
\usepackage{colortbl}
\usepackage{tcolorbox}
%\tcbset{enhanced,fonttitle=\bfseries\large,fontupper=\normalsize\sffamily, colback=yellow!10!white,colframe=red!50!black,colbacktitle=Salmon!30!white, coltitle=black,center title}

\begin{document}

\frame{\titlepage}

\section{Условный переход}

%%%%%%%%%%%%%%%%%%%%%%%%%%%%%%%%%%%%%%%%%%%%%%%%%%%%%%%%%%%%%%%%%%%%%%%%%%%%%%%
%%
%%
%%
\begin{frame}[fragile]\frametitle{Регистр флагов: флаги состояния}
  \begin{block}{\texttt{FLAGS}}
    \begin{center}
      \begin{tikzpicture}
        \node[bit, BitColorChange] (b0) at (0,0)                     {\texttt{CF}};
        \node[bit, BitColorHide, left=-\pgflinewidth of b0]  (b1)  {\texttt{1}};
        \node[bit, BitColorError, left=-\pgflinewidth of b1]  (b2)  {\texttt{PF}};
        \node[bit, BitColorHide, left=-\pgflinewidth of b2]  (b3)  {\texttt{0}};
        \node[bit, BitColorError, left=-\pgflinewidth of b3]  (b4)  {\texttt{AF}};
        \node[bit, BitColorHide, left=-\pgflinewidth of b4]  (b5)  {\texttt{0}};
        \node[bit, BitColorChange, left=-\pgflinewidth of b5]  (b6)  {\texttt{ZF}};
        \node[bit, BitColorChange, left=-\pgflinewidth of b6]  (b7)  {\texttt{SF}};
        \node[bit, left=0.5mm of b7]  (b8)  {\texttt{TF}};
        \node[bit, left=-\pgflinewidth of b8]  (b9)  {\texttt{IF}};
        \node[bit, left=-\pgflinewidth of b9]  (b10) {\texttt{DF}};
        \node[bit, BitColorChange, left=-\pgflinewidth of b10] (b11) {\texttt{OF}};
        \node[bit, left=-\pgflinewidth of b11] (b12) {};
        \node[bit, left=-\pgflinewidth of b12] (b13) {};
        \node[fill=Yellow500,draw=Yellow900] at (b13.east) {\texttt{IOPL}};
        \node[bit, left=-\pgflinewidth of b13] (b14) {\texttt{NT}};
        \node[bit, BitColorHide, left=-\pgflinewidth of b14] (b15) {\texttt{0}};
        \foreach \i in {0,...,15}
          \node[above=0cm of b\i] {\texttt{\scriptsize \i}};
      \end{tikzpicture}
    \end{center}
  \end{block}
  \begin{block}{}
    \begin{itemize}
      \item \textbf{CF} (Carry Flag) --- Флаг переноса
      \item PF (Parity Flag) --- Флаг чётности
      \item AF (Auxiliary Carry Flag) --- Вспомогательный флаг переноса
      \item \textbf{ZF} (Zero Flag) --- Флаг нуля
      \item \textbf{SF} (Sign Flag) --- Флаг знака
      \item \textbf{OF} (Overflow Flag) --- Флаг переполнения
    \end{itemize}
  \end{block}
\end{frame}

\begin{frame}[fragile]\frametitle{Регистр флагов и оператор SUB}
  \begin{block}{\texttt{FLAGS}}
    \begin{center}
      \begin{tikzpicture}
        \begin{scope}[scale=0.6, every node/.style={scale=0.6}]
          \node[bit] (CF) at (0,0)                   {\texttt{?}};
          \node[above=0cm of CF] (labelCF) {\texttt{CF}};
          \node[bit, left=-\pgflinewidth of CF] (ZF) {\texttt{?}};
          \node[above=0cm of ZF] (labelZF) {\texttt{ZF}};
          \node[bit, left=-\pgflinewidth of ZF] (SF) {\texttt{?}};
          \node[above=0cm of SF] (labelSF) {\texttt{SF}};
          \node[bit, left=-\pgflinewidth of SF] (OF) {\texttt{?}};
          \node[above=0cm of OF] (labelOF) {\texttt{OF}};
        \end{scope}
        \node[above left=0 of OF.south west] (labelFLAGS) {\texttt{FLAGS}};
        \node[below left=0.2cm and 0cm of labelFLAGS.south east] (labelEAX) {\texttt{EAX}};
        \node[databox=1, minimum width=3*\ByteWidth, text=brown, right=0cm of labelEAX] (EAX) {\only<1>{723}\only<2>{2000}\only<3>{1234}\only<4>{-2000000000}};
        \node[below left=0.2cm and 0cm of labelEAX.south east] (labelEBX) {\texttt{EBX}};
        \node[databox=1, minimum width=3*\ByteWidth, text=brown, right=0cm of labelEBX] (EBX) {\only<1>{1024}\only<2>{1200}\only<3>{1234}\only<4>{2000000000}};
        \node[state, fit=(CF) (labelOF) (labelFLAGS) (EAX) (EBX)] (State1) {};
%
        \begin{scope}[scale=0.6, every node/.style={scale=0.6}]
          \node<1>[bit, BitColorError] (CF) at (15cm,0) {\texttt{1}};
          \node<2,3,4>[bit] (CF) at (15cm,0) {\texttt{0}};
          \node[above=0cm of CF] (labelCF) {\texttt{CF}};
          \node<3>[bit, BitColorError, left=-\pgflinewidth of CF] (ZF) {\texttt{1}};
          \node<1,2,4>[bit, left=-\pgflinewidth of CF] (ZF) {\texttt{0}};
          \node[above=0cm of ZF] (labelZF) {\texttt{ZF}};
          \node<1>[bit, BitColorError, left=-\pgflinewidth of ZF] (SF) {\texttt{1}};
          \node<2,3,4>[bit, left=-\pgflinewidth of ZF] (SF) {\texttt{0}};
          \node[above=0cm of SF] (labelSF) {\texttt{SF}};
          \node<4>[bit, BitColorError, left=-\pgflinewidth of SF] (OF) {\texttt{1}};
          \node<1,2,3>[bit, left=-\pgflinewidth of SF] (OF) {\texttt{0}};
          \node[above=0cm of OF] (labelOF) {\texttt{OF}};
        \end{scope}
        \node[above left=0 of OF.south west] (labelFLAGS) {\texttt{FLAGS}};
        \node[below left=0.2cm and 0cm of labelFLAGS.south east] (labelEAX) {\texttt{EAX}};
        \node[databox=1, minimum width=3*\ByteWidth, text=brown, right=0cm of labelEAX] (EAX) {\only<1>{-301}\only<2>{800}\only<3>{0}\only<4>{294967296}};
        \node[below left=0.2cm and 0cm of labelEAX.south east] (labelEBX) {\texttt{EBX}};
        \node[databox=1, minimum width=3*\ByteWidth, text=brown, right=0cm of labelEBX] (EBX) {\only<1>{1024}\only<2>{1200}\only<3>{1234}\only<4>{2000000000}};
        \node[state, fit=(CF) (labelOF) (labelFLAGS) (EAX) (EBX)] (State2) {};
%
        \node[command] (cmd) at($(State1)!0.5!(State2)$) {\cl{sub eax, ebx}};
        \draw[->,myArrow] (State1) -- (cmd);
        \draw[->,myArrow] (cmd) -- (State2);
      \end{tikzpicture}
    \end{center}
  \end{block}
\end{frame}

\begin{frame}[fragile]\frametitle{Условный переход (1/2)}
  \begin{block}{Операторы перехода}
    \begin{minted}{nasm}
        JMP     метка ; безусловный переход на метку
        Jcc     метка ; переход на метку при выполнении условия сс
    \end{minted}
  \end{block}
  \begin{block}{}
    \begin{itemize}
      \item Операторы переходов не изменяют значения флагов, но все, кроме одной, срабатывают исходя только из значение флагов.
      \item Как правило, операторы условного перехода выполняется после арифметической или логической команды, или после специальной команды сравнения \cl{CMP}, \cl{TEST}.
      \item Условие перехода \cl{сс} можно представить как: \cl{[N]\{\{[G|L|A|B][E]\}|\{Z|S|C|O|P\}\}}, в соответствии со следующей таблицей:
    \end{itemize}
  \end{block}
\end{frame}

\begin{frame}\frametitle{Условный переход (2/2)}
  \begin{block}{}
    \begin{tcolorbox}[tabularx*={\arrayrulewidth0.5mm}{c|l|c|X},title={}]
Код & Name & Флаг & Название\\\hline\hline
N & Not & - & Отрицание условия (может комбинироваться со всеми)\\\hline
E & Equal & ZF=1 & Равно (может комбинироваться с G, L, A, B)\\\hline
G & Greater & ZF=0 и SF=OF & Больше (с учётом знака)\\\hline
L & Less & SF$\ne$OF & Меньше (с учётом знака)\\\hline
A & Above & CF=0 и ZF=0 & Выше\\\hline
B & Below & CF=1 & Ниже\\\hline
Z & Zero & ZF=1 & Ноль\\\hline
S & Sign & SF=1 & Отрицательное\\\hline
C & Carry & СF=1 & Перенос\\\hline
O & Overflow & OF=1 & Переполнение\\\hline
P & Parity & PF=1 & Паритет
    \end{tcolorbox}
  \end{block}
\end{frame}

\begin{frame}[fragile]\frametitle{Операторы CMP и TEST}
  \begin{block}{Операторый сравнения}
\begin{Verbatim}[commandchars=\\\{\}]
        \PYG{n+nf}{CMP}     \PYG{err}{Число1}\PYG{p}{,} \PYG{err}{Число2}
        \PYG{n+nf}{TEST}    \PYG{err}{Число1}\PYG{p}{,} \PYG{err}{Число2}
\end{Verbatim}
  \end{block}
  \begin{block}{}
    \begin{itemize}
      \item Команды \cl{CMP} и \cl{TEST}, как правило, используются перед командами условного перехода.
      \item Команда \cl{CMP} аналогична команде \cl{SUB}, но не сохраняет результат. Она удобна для сравнения чисел.
      \item Команда \cl{TEST} аналогична команде \cl{AND}, но не сохраняет результат. Она удобна для проверки битов.
    \end{itemize}
  \end{block}
\end{frame}

\begin{frame}\frametitle{Условный переход и CMP}
  \begin{block}{}
    \begin{tcolorbox}[tabularx*={\arrayrulewidth0.5mm}{c|X|c|c},title={}]
Мнемокод & Название & Условие & Значения флагов\\\hline\hline
\multicolumn{4}{c}{\textbf{Для всех чисел}}\\\hline
JE & Переход если равно & op1 $=$ op2 & ZF $=$ 1\\\hline
JNE & Переход если не равно & op1 $\ne$ op2 & ZF $=$ 0\\\hline
\multicolumn{4}{c}{\textbf{Для чисел со знаком}}\\\hline
JL/JNGE & Переход если меньше & op1 $<$ op2 & SF $\ne$ OF\\\hline
JLE/JNG & Переход если меньше или равно & op1 $\le$ op2 & SF $\ne$ OF или ZF $=$ 1\\\hline
JG/JNLE & Переход если больше & op1 $>$ op2 & SF $=$ OF и ZF $=$ 0\\\hline
JGE/JNL & Переход если больше или равно & op1 $\ge$ op2 & SF $=$ OF\\\hline
\multicolumn{4}{c}{\textbf{Для чисел без знака}}\\\hline
JB/JNAE & Переход если ниже & op1 $<$ op2 & CF $=$ 1\\\hline
JBE/JNA & Переход если ниже или равно & op1 $\le$ op2 & CF $=$ 1 или ZF $=$ 1\\\hline
JA/JNBE & Переход если выше & op1 $>$ op2 & CF $=$ 0 и ZF $=$ 0\\\hline
JAE/JNB & Переход если выше или равно & op1 $\ge$ op2 & CF $=$ 0
    \end{tcolorbox}
  \end{block}
\end{frame}

\begin{frame}[fragile]\frametitle{Пример <<Условный переход>>}
  \begin{block}{Что делает этот код?}
    \begin{minted}{nasm}
; ...
       mov     eax, [A]
       mov     ebx, [B]
       cmp     eax, ebx
       jge     m0
       mov     ecx, ebx
       jmp     m1
m0:
       mov     ecx, eax
m1:
;...
A:     dd      100
B:     dd      50
; ...
    \end{minted}
  \end{block}
\end{frame}

\begin{frame}[fragile]\frametitle{Пример <<Условный переход>>}
  \begin{columns}
    \begin{column}[]{0.35\textwidth}
      \begin{block}{Что делает этот код?}
        \begin{minted}{nasm}
; ...
C0:        mov     eax, [A]
C1:        mov     ebx, [B]
C2:        cmp     eax, ebx
C3:        jge     m0
C4:        mov     ecx, ebx
C5:        jmp     m1
    m0:
C6:        mov     ecx, eax
    m1:
; ...
    A:     dd      100
    B:     dd      50
; ...
        \end{minted}
      \end{block}
    \end{column}
    \begin{column}[]{0.55\textwidth}
      \begin{block}{}
        \begin{center}
          \begin{tikzpicture}
        \begin{scope}[scale=0.6, every node/.style={scale=0.6}]
          \node[bit] (CF1) at (0,0) {%
\only<1-2>{\texttt{?}}%
\only<3-4>{\texttt{0}}%
};
          \node[above=0cm of CF1] (labelCF1) {\texttt{CF}};
          \node[bit, left=-\pgflinewidth of CF1] (ZF1) {%
\only<1-2>{\texttt{?}}%
\only<3-4>{\texttt{0}}%
};
          \node[above=0cm of ZF1] (labelZF1) {\texttt{ZF}};
          \node[bit, left=-\pgflinewidth of ZF1] (SF1) {%
\only<1-2>{\texttt{?}}%
\only<3-4>{\texttt{0}}%
};
          \node[above=0cm of SF1] (labelSF1) {\texttt{SF}};
          \node[bit, left=-\pgflinewidth of SF1] (OF1) {%
\only<1-2>{\texttt{?}}%
\only<3-4>{\texttt{0}}%
};
          \node[above=0cm of OF1] (labelOF1) {\texttt{OF}};
        \end{scope}
        \node[above left=0 of OF1.south west] (labelFLAGS1) {\texttt{FLAGS}};
        \node[below left=0.2cm and 0cm of labelFLAGS1.south east] (labelECX1) {\texttt{ECX}};
        \node[databox=1, minimum width=3*\ByteWidth, text=brown, right=0cm of labelECX1] (ECX1) {%
\only<1-4>{?}%
};
        \node[databox=1, minimum width=3*\ByteWidth, text=brown, left=0cm of labelFLAGS1] (EAX1) {%
\only<1>{?}%
\only<2-4>{100}%
};
        \node[left=0cm of EAX1] (labelEAX1) {\texttt{EAX}};
        \node[below left=0.2cm and 0cm of labelEAX1.south east] (labelEBX1) {\texttt{EBX}};
        \node[databox=1, minimum width=3*\ByteWidth, text=brown, right=0cm of labelEBX1] (EBX1) {%
\only<1>{?}%
\only<2-4>{50}%
};
        \node[below left=0.2cm and 0cm of labelEBX1.south east] (labelEIP1) {\texttt{EIP}};
        \node[databox=1, minimum width=3*\ByteWidth, text=brown, right=0cm of labelEIP1] (EIP1) {%
\only<1>{C0}%
\only<2>{C2}%
\only<3>{C3}%
\only<4>{C6 (m0)}%
};
        \node[state, fit=(CF1) (labelOF1) (labelEAX1) (ECX1) (EIP1)] (State1) {};
%
        \begin{scope}[scale=0.6, every node/.style={scale=0.6}]
          \node<1>[bit] (CF2) at (0,-7cm) {\texttt{?}};
          \node<2>[bit, BitColorChange] (CF2) at (0,-7cm) {\texttt{0}};
          \node<3-4>[bit] (CF2) at (0,-7cm) {\texttt{0}};
          \node[above=0cm of CF2] (labelCF2) {\texttt{CF}};
          \node<1>[bit, left=-\pgflinewidth of CF2] (ZF2) {\texttt{?}};
          \node<2>[bit, left=-\pgflinewidth of CF2, BitColorChange] (ZF2) {\texttt{0}};
          \node<3-4>[bit, left=-\pgflinewidth of CF2] (ZF2) {\texttt{0}};
          \node[above=0cm of ZF2] (labelZF2) {\texttt{ZF}};
          \node<1>[bit, left=-\pgflinewidth of ZF2] (SF2) {\texttt{?}};
          \node<2>[bit, left=-\pgflinewidth of ZF2, BitColorChange] (SF2) {\texttt{0}};
          \node<3-4>[bit, left=-\pgflinewidth of ZF2] (SF2) {\texttt{0}};
          \node[above=0cm of SF2] (labelSF2) {\texttt{SF}};
          \node<1>[bit, left=-\pgflinewidth of SF2] (OF2) {\texttt{?}};
          \node<2>[bit, left=-\pgflinewidth of SF2, BitColorChange] (OF2) {\texttt{0}};
          \node<3-4>[bit, left=-\pgflinewidth of SF2] (OF2) {\texttt{0}};
          \node[above=0cm of OF2] (labelOF2) {\texttt{OF}};
        \end{scope}
        \node[above left=0 of OF2.south west] (labelFLAGS2) {\texttt{FLAGS}};
        \node[below left=0.2cm and 0cm of labelFLAGS2.south east] (labelECX2) {\texttt{ECX}};
        \node<1-3>[databox=1, minimum width=3*\ByteWidth, text=brown, right=0cm of labelECX2] (ECX2) {?};
        \node<4>[databox=1, minimum width=3*\ByteWidth, text=brown, right=0cm of labelECX2, BitColorChange] (ECX2) {100};
        \node<1>[databox=1, minimum width=3*\ByteWidth, text=brown, left=0cm of labelFLAGS2, BitColorChange] (EAX2) {100};
        \node<2-4>[databox=1, minimum width=3*\ByteWidth, text=brown, left=0cm of labelFLAGS2] (EAX2) {100};
        \node[left=0cm of EAX2] (labelEAX2) {\texttt{EAX}};
        \node[below left=0.2cm and 0cm of labelEAX2.south east] (labelEBX2) {\texttt{EBX}};
        \node<1>[databox=1, minimum width=3*\ByteWidth, text=brown, right=0cm of labelEBX2, BitColorChange] (EBX2) {50};
        \node<2-4>[databox=1, minimum width=3*\ByteWidth, text=brown, right=0cm of labelEBX2] (EBX2) {50};
        \node[below left=0.2cm and 0cm of labelEBX2.south east] (labelEIP2) {\texttt{EIP}};
        \node[databox=1, minimum width=3*\ByteWidth, text=brown, right=0cm of labelEIP2, BitColorChange] (EIP2) {%
\only<1>{C2}%
\only<2>{C3}%
\only<3>{C6 (m0)}%
\only<4>{C7 (m1)}%
};
        \node[state, fit=(CF2) (labelOF2) (labelEAX2) (ECX2) (EIP2)] (State2) {};
%
        \node[command, text width=2.5cm] (cmd) at($(State1.south)!0.5!(State2.north)$) {%
\only<1>{\cl{mov eax, [A]}\newline \cl{mov ebx, [B]}}%
\only<2>{\cl{cmp eax, ebx}}%
\only<3>{\cl{jge m0}}%
\only<4>{\cl{mov ecx, eax}}%
};
        \draw[->,myArrow] (State1) -- (cmd);
        \draw[->,myArrow] (cmd) -- (State2);
          \end{tikzpicture}
        \end{center}
      \end{block}
    \end{column}
  \end{columns}
\end{frame}

\begin{frame}[fragile]\frametitle{Пример <<Условный переход>>}
  \begin{columns}
    \begin{column}[]{0.45\textwidth}
      \begin{block}{Код}
        \begin{minted}{nasm}
; Старт
; Проверка условия
       cmp     eax, ebx
       jge     m0
; Список операций 2
       mov     ecx, ebx
       jmp     m1
m0:
; Список операций 1
       mov     ecx, eax
m1:
; Стоп
        \end{minted}
      \end{block}
    \end{column}
    \begin{column}[]{0.45\textwidth}
      \begin{block}{Блок-схема}
        \begin{center}
\begin{tikzpicture}[node distance=2cm, scale=0.8, every node/.style={scale=0.8}]
  \node (start) [startstop] {Старт};
  \node (dec1) [decision, below of=start] {Условие};
  \node (pro1) [process, below of=dec1, xshift=-2cm] {Список операций 1};
  \node (pro2) [process, below of=dec1, xshift=2cm] {Список операций 2};
  \node (stop) [startstop, below of=pro1, xshift=2cm] {Стоп};
  \draw [->,arrow] (start) -- (dec1);
  \draw [->,arrow] (dec1) -| node[anchor=south] {Да} (pro1);
  \draw [->,arrow] (dec1) -| node[anchor=south] {Нет} (pro2);
  \draw [->,arrow] (pro1) |- (stop);
  \draw [->,arrow] (pro2) |- (stop);
\end{tikzpicture}
        \end{center}
      \end{block}
    \end{column}
  \end{columns}
\end{frame}

\section{Циклы}

\begin{frame}[fragile]\frametitle{Пример}
  \begin{block}{Что делает этот код?}
    \begin{minted}{nasm}
; ...
        mov     ebx, 0
        push    dword 1
        push    str0
main_loop:
        call    printf
        mov     eax, ebx
        mov     ecx, [esp+4]
        add     eax, ecx
        mov     [esp+4], eax
        mov     ebx, ecx
        cmp     eax, [N]
        jb      main_loop
; ...
    \end{minted}
  \end{block}
\end{frame}

\end{document}
