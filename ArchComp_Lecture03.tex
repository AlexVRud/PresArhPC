\documentclass[pdf,9pt,aspectratio=169]{beamer}
\usepackage[utf8]{inputenc}

\usepackage{DejaVuSans}
\usepackage{DejaVuSerif}
\usepackage{DejaVuSansMono}
\usepackage[T2A]{fontenc}

\usepackage[russian]{babel}

\usepackage{hyperref}
\hypersetup{unicode=true,colorlinks=true}


\usetheme{Madrid}
\usefonttheme[stillsansserifsmall]{serif}
%\usefonttheme[onlylarge]{structurebold}
\usefonttheme[onlylarge]{structureitalicserif}

\title[Архитектура компьютера: Условный оператор]{Архитектура компьютера}
\subtitle{Условный оператор}
\author[А.В. Рудалёв]{Александр Васильевич Рудалёв}
\institute[ВШИТАС САФУ]{ВШИТАС САФУ}
\date[г. Архангельск, 2017 г.]{г. Архангельск, 2017 г.}

\usepackage{wrapfig}

\usepackage{NarfuLectures}

\usepackage{minted}
\usemintedstyle{default} 
\newcommand{\miln}[1]{\mintinline{nasm}{#1}}
\newcommand{\milt}[1]{\mintinline{text}{#1}}
\newcommand{\cl}[1]{\texttt{\bf #1}}

\begin{document}

\frame{\titlepage}

\section{Условный переход}

%%%%%%%%%%%%%%%%%%%%%%%%%%%%%%%%%%%%%%%%%%%%%%%%%%%%%%%%%%%%%%%%%%%%%%%%%%%%%%%
%%
%%
%%
\begin{frame}[fragile]\frametitle{Регистр флагов: флаги состояния}
  \begin{block}{\texttt{FLAGS}}
    \begin{center}
      \begin{tikzpicture}
        \node[bit, BitColorChange] (b0) at (0,0)                     {\texttt{CF}};
        \node[bit, BitColorHide, left=-\pgflinewidth of b0]  (b1)  {\texttt{1}};
        \node[bit, BitColorError, left=-\pgflinewidth of b1]  (b2)  {\texttt{PF}};
        \node[bit, BitColorHide, left=-\pgflinewidth of b2]  (b3)  {\texttt{0}};
        \node[bit, BitColorError, left=-\pgflinewidth of b3]  (b4)  {\texttt{AF}};
        \node[bit, BitColorHide, left=-\pgflinewidth of b4]  (b5)  {\texttt{0}};
        \node[bit, BitColorChange, left=-\pgflinewidth of b5]  (b6)  {\texttt{ZF}};
        \node[bit, BitColorChange, left=-\pgflinewidth of b6]  (b7)  {\texttt{SF}};
        \node[bit, left=0.5mm of b7]  (b8)  {\texttt{TF}};
        \node[bit, left=-\pgflinewidth of b8]  (b9)  {\texttt{IF}};
        \node[bit, left=-\pgflinewidth of b9]  (b10) {\texttt{DF}};
        \node[bit, BitColorChange, left=-\pgflinewidth of b10] (b11) {\texttt{OF}};
        \node[bit, left=-\pgflinewidth of b11] (b12) {};
        \node[bit, left=-\pgflinewidth of b12] (b13) {};
        \node[fill=Yellow500,draw=Yellow900] at (b13.east) {\texttt{IOPL}};
        \node[bit, left=-\pgflinewidth of b13] (b14) {\texttt{NT}};
        \node[bit, BitColorHide, left=-\pgflinewidth of b14] (b15) {\texttt{0}};
        \foreach \i in {0,...,15}
          \node[above=0cm of b\i] {\texttt{\scriptsize \i}};
      \end{tikzpicture}
    \end{center}
  \end{block}
  \begin{block}{}
    \begin{itemize}
      \item \textbf{CF} (Carry Flag) --- Флаг переноса
      \item PF (Parity Flag) --- Флаг чётности
      \item AF (Auxiliary Carry Flag) --- Вспомогательный флаг переноса
      \item \textbf{ZF} (Zero Flag) --- Флаг нуля
      \item \textbf{SF} (Sign Flag) --- Флаг знака
      \item \textbf{OF} (Overflow Flag) --- Флаг переполнения
    \end{itemize}
  \end{block}
\end{frame}

\begin{frame}[fragile]\frametitle{Регистр флагов и команда SUB}
\end{frame}

\begin{frame}[fragile]\frametitle{Условный переход}
\end{frame}

\end{document}
