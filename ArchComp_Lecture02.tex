\documentclass[pdf,9pt,aspectratio=169]{beamer}
\usepackage[utf8]{inputenc}

\usepackage{DejaVuSans}
\usepackage{DejaVuSerif}
\usepackage{DejaVuSansMono}
\usepackage[T2A]{fontenc}

\usepackage[russian]{babel}

\usepackage{hyperref}
\hypersetup{unicode=true}

\usetheme{Madrid}
\usefonttheme[stillsansserifsmall]{serif}
%\usefonttheme[onlylarge]{structurebold}
\usefonttheme[onlylarge]{structureitalicserif}

\title[Архитектура компьютера: Целочисленная арифметика]{Архитектура компьютера}
\subtitle{Целочисленная арифметика}
\author[А.В. Рудалёв]{Александр Васильевич Рудалёв}
\institute[ВШИТАС САФУ]{ВШИТАС САФУ}
\date[г. Архангельск, 2017 г.]{г. Архангельск, 2017 г.}

\usepackage{wrapfig}

\usepackage{NarfuLectures}

\usepackage{minted}
\usemintedstyle{default} 
\newcommand{\miln}[1]{\mintinline{nasm}{#1}}
\newcommand{\milt}[1]{\mintinline{text}{#1}}
\newcommand{\cl}[1]{\texttt{\bf #1}}

\begin{document}

\frame{\titlepage}

\section{Операции умножения и деления}

%%%%%%%%%%%%%%%%%%%%%%%%%%%%%%%%%%%%%%%%%%%%%%%%%%%%%%%%%%%%%%%%%%%%%%%%%%%%%%%
%%
%%
%%
\begin{frame}[fragile]\frametitle{Переполнение}
  \begin{columns}[T]
    \begin{column}[]{0.45\textwidth}  
      \begin{exampleblock}<1->{Ok}
        \begin{center}
          \begin{tikzpicture}
            \node[] (h1) at (0,0) {\texttt{0x00004B22\textsubscript{16}}};
            \node[below left=0cm and 0cm of h1.south east] (h2) {\texttt{0x00009A8C\textsubscript{16}}};
            \node[below left=0cm and 0cm of h2.south east] (hr) {\texttt{  0x2D5B8A98\textsubscript{16}}};
            \node[right=0cm of h1, text=Blue700] (d1) {\texttt{(19234\textsubscript{10})}};
            \node[right=0cm of h2, text=Blue700] (d2) {\texttt{(39564\textsubscript{10})}};
            \node[right=0cm of hr, text=Green700] (dr) {\texttt{(760973976\textsubscript{10})}};
            \node[anchor=east] at (h1.south west) {$\times$};
            \draw[] (hr.north east) -- (hr.north west);
          \end{tikzpicture}
        \end{center}
      \end{exampleblock}
    \end{column}
    \begin{column}[]{0.45\textwidth}  
      \begin{alertblock}<2->{Error}
        \begin{center}
          \begin{tikzpicture}
            \node[] (h1) at (0,0) {\texttt{0x0000E762\textsubscript{16}}};
            \node[below left=0cm and 0cm of h1.south east] (h2) {\texttt{0x00009A8C\textsubscript{16}}};
            \node[below left=0cm and 0cm of h2.south east] (hr) {\texttt{  0x8BAF7D98\textsubscript{16}}};
            \node[right=0cm of h1, text=Blue700] (d1) {\texttt{(59234\textsubscript{10})}};
            \node[right=0cm of h2, text=Blue700] (d2) {\texttt{(39564\textsubscript{10})}};
            \node[right=0cm of hr, text=Red700] (dr) {\texttt{(-1951433320\textsubscript{10})}};
            \node[anchor=east] at (h1.south west) {$\times$};
            \draw[] (hr.north east) -- (hr.north west);
          \end{tikzpicture}
        \end{center}
      \end{alertblock}
    \end{column}
  \end{columns}
\hspace{1cm}
  \begin{columns}[T]
    \begin{column}[]{0.45\textwidth}  
      \begin{block}<3->{Вопросы}
        \begin{itemize}
          \item Как узнать о переполнении?
          \item Как исправить?
          \item Как воспользоваться?
       \end{itemize}
     \end{block}
    \end{column}
  \end{columns}
\end{frame}

\end{document}
