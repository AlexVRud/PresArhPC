\documentclass[pdf,9pt,aspectratio=169]{beamer}
\usepackage[utf8]{inputenc}

\usepackage{DejaVuSans}
\usepackage{DejaVuSerif}
\usepackage{DejaVuSansMono}
\usepackage[T2A]{fontenc}

\usepackage[russian]{babel}

\usepackage{hyperref}
\hypersetup{unicode=true}

\usetheme{Madrid}
\usefonttheme[stillsansserifsmall]{serif}
%\usefonttheme[onlylarge]{structurebold}
\usefonttheme[onlylarge]{structureitalicserif}

\title[Архитектура компьютера: Целочисленная арифметика]{Архитектура компьютера}
\subtitle{Целочисленная арифметика}
\author[А.В. Рудалёв]{Александр Васильевич Рудалёв}
\institute[ВШИТАС САФУ]{ВШИТАС САФУ}
\date[г. Архангельск, 2017 г.]{г. Архангельск, 2017 г.}

\usepackage{wrapfig}

\usepackage{NarfuLectures}

\usepackage{minted}
\usemintedstyle{default} 
\newcommand{\miln}[1]{\mintinline{nasm}{#1}}
\newcommand{\milt}[1]{\mintinline{text}{#1}}
\newcommand{\cl}[1]{\texttt{\bf #1}}

\begin{document}

\frame{\titlepage}

\section{Операции умножения и деления}

%%%%%%%%%%%%%%%%%%%%%%%%%%%%%%%%%%%%%%%%%%%%%%%%%%%%%%%%%%%%%%%%%%%%%%%%%%%%%%%
%%
%%
%%
\begin{frame}[fragile]\frametitle{Переполнение}
  \begin{columns}[T]
    \begin{column}[]{0.45\textwidth}  
      \begin{exampleblock}<1->{Ok}
        \begin{center}
          \begin{tikzpicture}
            \node[] (h1) at (0,0) {\texttt{0x00004B22\textsubscript{16}}};
            \node[below left=0cm and 0cm of h1.south east] (h2) {\texttt{0x00009A8C\textsubscript{16}}};
            \node[below left=0cm and 0cm of h2.south east] (hr) {\texttt{  0x2D5B8A98\textsubscript{16}}};
            \node[right=0cm of h1, text=Blue700] (d1) {\texttt{(19234\textsubscript{10})}};
            \node[right=0cm of h2, text=Blue700] (d2) {\texttt{(39564\textsubscript{10})}};
            \node[right=0cm of hr, text=Green700] (dr) {\texttt{(760973976\textsubscript{10})}};
            \node[anchor=east] at (h1.south west) {$\times$};
            \draw[] (hr.north east) -- (hr.north west);
          \end{tikzpicture}
        \end{center}
      \end{exampleblock}
    \end{column}
    \begin{column}[]{0.45\textwidth}  
      \begin{alertblock}<2->{Error}
        \begin{center}
          \begin{tikzpicture}
            \node[] (h1) at (0,0) {\texttt{0x0000E762\textsubscript{16}}};
            \node[below left=0cm and 0cm of h1.south east] (h2) {\texttt{0x00009A8C\textsubscript{16}}};
            \node[below left=0cm and 0cm of h2.south east] (hr) {\texttt{  0x8BAF7D98\textsubscript{16}}};
            \node[right=0cm of h1, text=Blue700] (d1) {\texttt{(59234\textsubscript{10})}};
            \node[right=0cm of h2, text=Blue700] (d2) {\texttt{(39564\textsubscript{10})}};
            \node[right=0cm of hr, text=Red700] (dr) {\texttt{(-1951433320\textsubscript{10})}};
            \node[anchor=east] at (h1.south west) {$\times$};
            \draw[] (hr.north east) -- (hr.north west);
          \end{tikzpicture}
        \end{center}
      \end{alertblock}
    \end{column}
  \end{columns}
\hspace{1cm}
  \begin{columns}[T]
    \begin{column}[]{0.45\textwidth}  
      \begin{block}<3->{Вопросы}
        \begin{itemize}
          \item Как узнать о переполнении?
          \item Как исправить?
          \item Как воспользоваться?
       \end{itemize}
     \end{block}
    \end{column}
  \end{columns}
\end{frame}

\begin{frame}[fragile]\frametitle{Регистр флагов}
  \begin{block}{\texttt{FLAGS}}
    \begin{center}
      \begin{tikzpicture}
        \node[bit, BitColorError] (b0) at (0,0)                     {\texttt{CF}};
        \node[bit, BitColorHide, left=-\pgflinewidth of b0]  (b1)  {\texttt{1}};
        \node[bit, left=-\pgflinewidth of b1]  (b2)  {\texttt{PF}};
        \node[bit, BitColorHide, left=-\pgflinewidth of b2]  (b3)  {\texttt{0}};
        \node[bit, left=-\pgflinewidth of b3]  (b4)  {\texttt{AF}};
        \node[bit, BitColorHide, left=-\pgflinewidth of b4]  (b5)  {\texttt{0}};
        \node[bit, left=-\pgflinewidth of b5]  (b6)  {\texttt{ZF}};
        \node[bit, left=-\pgflinewidth of b6]  (b7)  {\texttt{SF}};
        \node[bit, left=0.5mm of b7]  (b8)  {\texttt{TF}};
        \node[bit, left=-\pgflinewidth of b8]  (b9)  {\texttt{IF}};
        \node[bit, left=-\pgflinewidth of b9]  (b10) {\texttt{DF}};
        \node[bit, BitColorError, left=-\pgflinewidth of b10] (b11) {\texttt{OF}};
        \node[bit, left=-\pgflinewidth of b11] (b12) {};
        \node[bit, left=-\pgflinewidth of b12] (b13) {};
        \node[fill=Yellow500,draw=Yellow900] at (b13.east) {\texttt{IOPL}};
        \node[bit, left=-\pgflinewidth of b13] (b14) {\texttt{NT}};
        \node[bit, BitColorHide, left=-\pgflinewidth of b14] (b15) {\texttt{0}};
        \foreach \i in {0,...,15}
          \node[above=0cm of b\i] {\texttt{\scriptsize \i}};
      \end{tikzpicture}
    \end{center}
  \end{block}
  \begin{block}{}
    \begin{itemize}
      \item CF (Carry Flag) --- Флаг переноса
      \item OF (Overflow Flag) --- Флаг переполнения
    \end{itemize}
  \end{block}
\end{frame}

\begin{frame}[fragile]\frametitle{Результат двойного размера}
  \begin{exampleblock}{Ok}
    \begin{center}
      \begin{tikzpicture}
        \node[] (h1) at (0,0) {\texttt{0x0001E240\textsubscript{16}}};
        \node[below left=0cm and 0cm of h1.south east] (h2) {\texttt{0x000C0A14\textsubscript{16}}};
        \node[below left=0cm and 0cm of h2.south east] (hr) {\texttt{  0x00000016 0xADFC2D00\textsubscript{16}}};
        \node[right=0cm of h1, text=Blue700] (d1) {\texttt{(123456\textsubscript{10})}};
        \node[right=0cm of h2, text=Blue700] (d2) {\texttt{(789012\textsubscript{10})}};
        \node[right=0cm of hr, text=Green700] (dr) {\texttt{(97408265472\textsubscript{10})}};
        \node[anchor=east] at (h1.south west) {$\times$};
        \draw[] (hr.north east) -- (hr.north west);
        \node[bit, BitColorError,right=5mm of dr] (of) {\texttt{1}};
        \node[bit, BitColorError,right=1mm of of] (cf) {\texttt{1}};
        \node[above=0cm of of] {\texttt{\scriptsize of}};
        \node[above=0cm of cf] {\texttt{\scriptsize cf}};
      \end{tikzpicture}
    \end{center}
  \end{exampleblock}
  \begin{exampleblock}<2->{}
    \begin{minted}{nasm}
        mov     eax, 0x0001E240
        mov     ecx, 0x000C0A14
        mul     ecx
; edx == 0x00000016 , eax == 0xADFC2D00
; edx:eax == 0x00000016ADFC2D00
    \end{minted}
  \end{exampleblock}
\end{frame}

\begin{frame}[fragile]\frametitle{Применение: длинная арифметика}
  \begin{block}<1->{Определение [wiki]}
\textbf{Длинная арифметика} --- выполняемые с помощью вычислительной машины арифметические операции (сложение, вычитание, умножение, деление, возведение в степень, элементарные функции) над числами, разрядность которых превышает длину машинного слова данной вычислительной машины.
  \end{block}

  \begin{block}<2->{Применение}
Одна из областей применения длинной арифметики: криптография, где используются 512-разрядный, 1024-разрядные и более длинные целые числа.
  \end{block}

  \begin{block}<3->{Почитать}
    \begin{enumerate}
      \item Алгоритм шифрования RSA на пальцах // Technology Box URL: \url{http://teh-box.ru/informationsecurity/algoritm-shifrovaniya-rsa-na-palcax.html} (дата обращения: 02.03.2017).
    \end{enumerate}
  \end{block}
\end{frame}

\begin{frame}[fragile]\frametitle{Умножение: команда IMUL}
\end{frame}

\begin{frame}[fragile]\frametitle{Умножение: команда MUL}
\end{frame}

\begin{frame}[fragile]\frametitle{Деление}
\end{frame}

\section{Длинная арифметика: первый взгляд}

\begin{frame}[fragile]\frametitle{Как сложить ОЧЕНЬ большие числа}
\end{frame}

\begin{frame}[fragile]\frametitle{Сложение с переносом: команда ADC}
\end{frame}

\begin{frame}[fragile]\frametitle{Быстрое умножение}
\end{frame}

\begin{frame}[fragile]\frametitle{Что ещё}
\end{frame}

\end{document}
