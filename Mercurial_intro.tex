\documentclass[pdf,9pt,aspectratio=169]{beamer}
\usepackage[utf8]{inputenc}

\AtBeginSection[]{
  \begin{frame}
  \vfill
  \centering
  \begin{beamercolorbox}[sep=8pt,center,shadow=true,rounded=true]{title}
    \usebeamerfont{title}\secname\par%
  \end{beamercolorbox}
  \vfill
  \end{frame}
}

\usepackage{DejaVuSans}
\usepackage{DejaVuSerif}
\usepackage{DejaVuSansMono}
\usepackage[T2A]{fontenc}

\usepackage[russian]{babel}

\usepackage{hyperref}
\hypersetup{unicode=true}

\usetheme{Madrid}
\usefonttheme[stillsansserifsmall]{serif}
%\usefonttheme[onlylarge]{structurebold}
\usefonttheme[onlylarge]{structureitalicserif}

\title[Распределённые системы котроля версий: Mercurial]{Распределённые системы котроля версий}
\subtitle{Mercurial}
\author[А.В. Рудалёв]{Александр Васильевич Рудалёв}
\institute[ИМИКТ САФУ]{ИМИКТ САФУ}
\date[г. Архангельск, 2016 г.]{г. Архангельск, 2016 г.}

\usepackage{wrapfig}
\usepackage{color}
\usepackage{xcolor}

\usepackage{tikz}
\usetikzlibrary{arrows}
\usetikzlibrary{arrows.meta}
\usetikzlibrary{babel}
\usetikzlibrary{calc}
\usetikzlibrary{decorations.pathmorphing}
\usetikzlibrary{decorations.pathreplacing}
\usetikzlibrary{fit}
\usetikzlibrary{positioning}
\usetikzlibrary{shapes}
\usetikzlibrary{shapes.geometric}
\usetikzlibrary{shapes.multipart}
\usetikzlibrary{topaths}

\tikzset{
  MyChar/.style={
    rectangle, rounded corners=1mm,
    minimum height=6mm,
    minimum width=4.8mm,
    very thick, draw=black!50,
    top color=white,bottom color=black!20,
  },
  startstop/.style={rectangle, rounded corners=0.4cm, minimum width=3cm, minimum height=1cm,text centered, draw=black, fill=red!30},
  io/.style={trapezium, trapezium left angle=70, trapezium right angle=110, minimum width=3cm, minimum height=1cm, text centered, draw=black, fill=blue!30},
  process/.style={rectangle, minimum width=3cm, minimum height=1cm, text centered, draw=black, fill=orange!30},
  decision/.style={diamond, minimum width=3cm, minimum height=1cm, text centered, draw=black, fill=green!30},
  arrow/.style={thick,->,>=stealth},
  file/.style={rectangle, draw=black, fill=white, minimum width=2cm, minimum height=1cm},
  program/.style={rectangle, draw=black, fill=blue!20!white, rounded corners},
  recept/.style={
    draw,
    align=left,
    text width=2cm,
  },
}

\usepackage{minted}
\usemintedstyle{default} 
\newcommand{\cpil}[1]{\mintinline{python3}{#1}}

\begin{document}

\frame{\titlepage}

%%%%%%%%%%%%%%%%%%%%%%%%%%%%%%%%%%%%%%%%%%%%%%%%%%%%%%%%%%%%%%%%%%%%%%%%%%%%%%%
%%
%% Постановка проблемы
%%
\begin{frame}{Постановка проблемы}
  \begin{exampleblock}{Развитие проекта}
    \begin{center}
\begin{tikzpicture}[node distance=3.5cm]
\fill[white,draw=gray, dashed] (-1.75cm,1cm) -- (12cm,1cm) -- (12cm,-1.75cm) -- (-1.75cm,-1.75cm) -- cycle;
\node[below right, gray] at (-1.75cm,1cm) {Данные};
\fill[white,draw=gray, dashed] (-1.75cm,-2cm) -- (12cm,-2cm) -- (12cm,-4cm) -- (-1.75cm,-4cm) -- cycle;
\node[above right, gray] at (-1.75cm,-4cm) {Действия};

\node<1->[recept, align=center, fill=green!50] (p1t) {Меню};
\node<1->[recept, fill=green!20, below=-1pt of p1t] (p1milk) {Молоко};

\node<3->[recept, align=center, fill=green!50, right of=p1t] (p2t) {Меню};
\node<3->[recept, fill=green!20, below=-1pt of p2t] (p2milk) {Молоко};
\node<3->[recept, fill=blue!20, below=-1pt of p2milk] (p2soup) {Суп};

\node<5->[recept, align=center, fill=green!50, right of=p2t] (p3t) {Меню};
\node<5->[recept, fill=green!20, below=-1pt of p3t] (p3milk) {Молоко};
\node<5->[recept, fill=blue!20, below=-1pt of p3milk] (p3soup) {Суп};
\node<5->[recept, fill=red!20, below=-1pt of p3soup] (p3omelette) {Омлет};

\node<7->[recept, align=center, fill=green!50, right of=p3t] (p4t) {Меню};
\node<7->[recept, fill=yellow!20, below=-1pt of p4t] (p4tee) {Чай};
\node<7->[recept, fill=blue!20, below=-1pt of p4tee] (p4soup) {Суп};
\node<7->[recept, fill=red!20, below=-1pt of p4soup] (p4omelette) {Омлет};

\node<2->[recept, rounded corners, fill=blue!20, below] (p1to2soup) at (1.75cm, -2.5cm) {+Суп};
\node<4->[recept, rounded corners, fill=red!20, below] (p2to3omelette) at (5.25cm, -2.5cm)  {+Омлет};
\node<6->[rectangle split, align=left, rectangle split parts=2, rectangle split part fill={green!20, yellow!20}, draw, text width =2cm, rounded corners, below] (p3to4) at (8.75cm, -2.5cm)  {-Молоко \nodepart{two} +Чай};

\draw<2->[-o] (p1milk) to[out=-90,in=180] (p1to2soup);
\draw<3->[->] (p1to2soup) to[out=0,in=-90] ($(p2soup.south)+(-0.2cm,0)$);
\draw<4->[-o] ($(p2soup.south)+(0.2cm,0)$) to[out=-90,in=180] (p2to3omelette);
\draw<5->[->] (p2to3omelette) to[out=0,in=-90] ($(p3omelette.south)+(-0.2cm,0)$);
\draw<6->[-o] ($(p3omelette.south)+(0.2cm,0)$) to[out=-90,in=180] (p3to4);
\draw<7->[->] (p3to4) to[out=0,in=-90] ($(p4omelette.south)$);

\draw[->, draw, thick] (-2.25cm,-1.875cm) -- (12.5cm,-1.875cm) node[below left] {\textit{время} $t$};

\end{tikzpicture}
    \end{center}
  \end{exampleblock}
\end{frame}

%%%%%%%%%%%%%%%%%%%%%%%%%%%%%%%%%%%%%%%%%%%%%%%%%%%%%%%%%%%%%%%%%%%%%%%%%%%%%%%
%%
%% Постановка проблемы
%%
\begin{frame}{Постановка проблемы}
  \begin{exampleblock}{Развитие проекта}
    \begin{center}
\begin{tikzpicture}[node distance=3.5cm]
\node[recept, align=center, fill=green!50] (p1t) {Меню};
\node[recept, fill=green!20, below=-1pt of p1t] (p1milk) {Молоко};

\node[recept, rounded corners, fill=blue!20, right of=p1t] (p1to2soup) {+Суп};
\node[recept, rounded corners, fill=red!20, below=2pt of p1to2soup] (p2to3omelette) {+Омлет};
\node[rectangle split, align=left, rectangle split parts=2, rectangle split part fill={green!20, yellow!20}, draw, text width =2cm, rounded corners, below=2pt of p2to3omelette] (p3to4) {-Молоко \nodepart{two} +Чай};

\node[recept, align=center, fill=green!50, right of=p1to2soup] (p4t) {Меню};
\node[recept, fill=yellow!20, below=-1pt of p4t] (p4tee) {Чай};
\node[recept, fill=blue!20, below=-1pt of p4tee] (p4soup) {Суп};
\node[recept, fill=red!20, below=-1pt of p4soup] (p4omelette) {Омлет};

\draw[-o] (p1t) to[out=0,in=180] (p1to2soup);
\draw[->] (p1to2soup) to[out=0,in=0] (p2to3omelette);
\draw[->] (p2to3omelette) to[out=180,in=180] (p3to4);
\draw[->] (p3to4) to[out=0,in=180] (p4t);
\end{tikzpicture}
    \end{center}
  \end{exampleblock}
\end{frame}

%%%%%%%%%%%%%%%%%%%%%%%%%%%%%%%%%%%%%%%%%%%%%%%%%%%%%%%%%%%%%%%%%%%%%%%%%%%%%%%
%%
%% Постановка проблемы
%%
\begin{frame}{Постановка проблемы}
  \begin{exampleblock}{Развитие проекта}
    \begin{center}
\begin{tikzpicture}[node distance=3.5cm]
\fill[white,draw=gray, dashed] (-1.75cm,0.75cm) -- (12cm,0.75cm) -- (12cm,-4.75cm)
   -- (5.5cm,-4.75cm) -- (5.5cm, -1.25cm) -- (-1.75cm, -1.25cm) -- cycle;
\node[below right, gray] at (-1.75cm,0.75cm) {На работе};

\fill[white,draw=gray, dashed] (-1.75cm,-1.35cm) -- (5.4cm,-1.35cm) -- (5.4cm,-4.75cm) -- (-1.75cm,-4.75cm) -- cycle;
\node[above right, gray] at (-1.75cm,-4.75cm) {Дома};

\node<1->[recept, align=center, fill=green!50] (p1t) {Меню};
\node<1->[recept, fill=green!20, below=-1pt of p1t] (p1milk) {Молоко};

\node<2->[recept, rounded corners, fill=white, below=1cm of p1milk] (p1copy) {\small Копирование};

\node<2->[recept, align=center, fill=green!50, below=1cm of p1copy] (p21t) {Меню};
\node<2->[recept, fill=green!20, below=-1pt of p21t] (p21milk) {Молоко};

\node<3->[recept, rounded corners, fill=red!20, right of=p21t] (p2to3omelette) {+Омлет};
\node<3->[rectangle split, align=left, rectangle split parts=2, rectangle split part fill={green!20, yellow!20}, draw, text width =2cm, rounded corners, below=2pt of p2to3omelette] (p3to4) {-Молоко \nodepart{two} +Чай};

\node<4->[recept, rounded corners, fill=blue!20, right=4cm of p1t] (p1to2soup) {+Суп};

\node<5->[recept, rounded corners, fill=white, right=5cm of p1copy] (p3merge) {\small Слияние};

\node<6->[recept, align=center, fill=green!50, right of=p3merge, yshift=2cm] (p4t) {Меню};
\node<6->[recept, fill=yellow!20, below=-1pt of p4t] (p4tee) {Чай};
\node<6->[recept, fill=blue!20, below=-1pt of p4tee] (p4soup) {Суп};
\node<6->[recept, fill=red!20, below=-1pt of p4soup] (p4omelette) {Омлет};

\node<6->[recept, align=center, fill=green!50, below=1.4cm of p4soup] (p24t) {Меню};
\node<6->[recept, fill=yellow!20, below=-1pt of p24t] (p24tee) {Чай};
\node<6->[recept, fill=red!20, below=-1pt of p24tee] (p24omelette) {Омлет};
\node<6->[recept, fill=blue!20, below=-1pt of p24omelette] (p24soup) {Суп};

\draw<2->[-o] (p1milk) -- (p1copy);
\draw<2->[->] (p1copy) -- (p21t);

\draw<3->[-o] (p21t) -- (p2to3omelette);
\draw<3->[->] (p2to3omelette) to[out=190,in=180] (p3to4);

\draw<4->[-o] (p1t) -- (p1to2soup);

\draw<5->[-o] (p1to2soup) to[out=-90,in=90] (p3merge);
\draw<5->[-o] (p3to4) to[out=0,in=-90] (p3merge);

\draw<6->[->] (p3merge) to[out=10,in=180] node[above left]{?} (p4t);
\draw<6->[->] (p3merge) to[out=-10,in=180] node[above]{?} (p24t);

\end{tikzpicture}
    \end{center}
  \end{exampleblock}
\end{frame}


%%%%%%%%%%%%%%%%%%%%%%%%%%%%%%%%%%%%%%%%%%%%%%%%%%%%%%%%%%%%%%%%%%%%%%%%%%%%%%%
%%
%% Определения
%%
\begin{frame}\frametitle{Введение в предмет <<Архитектура компьютера>>}
  \begin{block}<1->{Определение}
    \textbf{Архитектура компьютера} "---  это описание его организации и принципов функционирования его структурных элементов. Включает основные устройства ЭВМ и структуру связей между ними.
  \end{block}
  \begin{block}<2->{О чём пойдёт речь}
    В рамках дисциплины мы должны понять как работает компьютер и программное обеспечение на нём.
  \end{block}
\end{frame}

%%%%%%%%%%%%%%%%%%%%%%%%%%%%%%%%%%%%%%%%%%%%%%%%%%%%%%%%%%%%%%%%%%%%%%%%%%%%%%%
%%
%% Содержание дисциплины
%%
\begin{frame}\frametitle{Содержание дисциплины}
  \begin{block}<1->{}
    \begin{itemize}
      \item Основы архитектуры ЭВМ (базовые части и их связи).
      \item Язык программирования Ассемблер (язык процессора).
      \item Компиляция, загрузка и выполнение программы.
      \item Инструменты программиста.
    \end{itemize}
  \end{block}
\end{frame}

%%%%%%%%%%%%%%%%%%%%%%%%%%%%%%%%%%%%%%%%%%%%%%%%%%%%%%%%%%%%%%%%%%%%%%%%%%%%%%%
%%
%% Введение в архитектуру ЭВМ
%%
\section{Введение в архитектуру ЭВМ}


%%%%%%%%%%%%%%%%%%%%%%%%%%%%%%%%%%%%%%%%%%%%%%%%%%%%%%%%%%%%%%%%%%%%%%%%%%%%%%%
%%
%% Процесс компиляции
%%
\begin{frame}{Процесс компиляции}
  \begin{exampleblock}<1->{Компиляция программы на ЯП Си: два этапа}
    \begin{center}
\begin{tikzpicture}[node distance=2.5cm]
  \node [file, fill=green!20!white] (main_c) at (0,0) {main.c};
  \node [program, right of=main_c] (gcc1) {gcc.exe};
  \node [file, fill=yellow!20!white, right of=gcc1] (main_obj) {main.o};
  \node [file, fill=yellow!20!white, below=1cm of main_obj, xshift=2.5cm] (other_obj) {other.o};
  \node [program, right of=main_obj] (ld) {ld.exe};
  \node [file, fill=red!20!white, right of=ld] (mainexe) {main.exe};
  \node [file, fill=red!20!white, below=1cm of mainexe, xshift=2.5cm] (other_dll) {other.dll};
  \draw [-o] (main_c) -- (gcc1);
  \draw [->] (gcc1) -- (main_obj);
  \draw [-o] (main_obj) -- (ld);
  \draw [-o] (other_obj) -- (ld);
  \draw [->] (ld) -- (mainexe);
  \draw [->, dashed] (mainexe) -| (other_dll);
  \draw [->, dashed] (other_obj) -- (other_dll);
  \node [dashed, draw=blue, fit=(main_c) (main_obj), yshift=-0.02cm] (compiling) {};
  \node [rectangle, rounded corners=0.2cm, draw=blue, fill=white,above=0.1cm of compiling] () {Компиляция};
  \node [dashed, draw=red, fit=(main_obj) (other_obj) (mainexe), yshift=0.02cm] (linking) {};
  \node [rectangle, rounded corners=0.2cm, draw=red, fill=white,above=0.1cm of linking] () {Компоновка};
\end{tikzpicture}
    \end{center}
  \end{exampleblock}
\end{frame}

%%%%%%%%%%%%%%%%%%%%%%%%%%%%%%%%%%%%%%%%%%%%%%%%%%%%%%%%%%%%%%%%%%%%%%%%%%%%%%%
%%
%% 
%%
\begin{frame}{}
  \vfill
  \begin{beamercolorbox}[sep=8pt,center,shadow=true,rounded=true]{title}
    \usebeamerfont{title}Вопросы?
  \end{beamercolorbox}
  \vfill
  \begin{columns}[T]
    \begin{column}[]{0.45\textwidth}  
      \begin{exampleblock}<1->{Сделанов в}
        \begin{center}
           \Huge\LaTeXe
        \end{center}
      \end{exampleblock}
    \end{column}
    \begin{column}[]{0.45\textwidth}  
      \begin{block}<1->{Использовано}
        \begin{itemize}
          \item пакеты: beamer, tikz
          \item строк кода: >150 
        \end{itemize}
      \end{block}
    \end{column}
  \end{columns}
  \vfill
\end{frame}

\end{document}
