\documentclass[pdf,9pt,aspectratio=169]{beamer}
\usepackage[utf8]{inputenc}

\AtBeginSection[]{
  \begin{frame}
  \vfill
  \centering
  \begin{beamercolorbox}[sep=8pt,center,shadow=true,rounded=true]{title}
    \usebeamerfont{title}\secname\par%
  \end{beamercolorbox}
  \vfill
  \end{frame}
}

\usepackage{DejaVuSans}
\usepackage{DejaVuSerif}
\usepackage{DejaVuSansMono}
\usepackage[T2A]{fontenc}

\usepackage[russian]{babel}

\usepackage{hyperref}
\hypersetup{unicode=true}

\usetheme{Madrid}
\usefonttheme[stillsansserifsmall]{serif}
%\usefonttheme[onlylarge]{structurebold}
\usefonttheme[onlylarge]{structureitalicserif}

\title[Системы управления версиями: Mercurial]{Системы управления версиями}
\subtitle{Mercurial}
\author[А.В. Рудалёв]{Александр Васильевич Рудалёв}
\institute[ИМИКТ САФУ]{ИМИКТ САФУ}
\date[г. Архангельск, 2016 г.]{г. Архангельск, 2016 г.}

\usepackage{wrapfig}
\usepackage{color}
\usepackage{xcolor}

\usepackage{tikz}
\usetikzlibrary{arrows}
\usetikzlibrary{arrows.meta}
\usetikzlibrary{babel}
\usetikzlibrary{calc}
\usetikzlibrary{decorations.pathmorphing}
\usetikzlibrary{decorations.pathreplacing}
\usetikzlibrary{fit}
\usetikzlibrary{patterns}
\usetikzlibrary{positioning}
\usetikzlibrary{shapes}
\usetikzlibrary{shapes.geometric}
\usetikzlibrary{shapes.multipart}
\usetikzlibrary{topaths}

\tikzset{
  MyChar/.style={
    rectangle, rounded corners=1mm,
    minimum height=6mm,
    minimum width=4.8mm,
    very thick, draw=black!50,
    top color=white,bottom color=black!20,
  },
  startstop/.style={rectangle, rounded corners=0.4cm, minimum width=3cm, minimum height=1cm,text centered, draw=black, fill=red!30},
  io/.style={trapezium, trapezium left angle=70, trapezium right angle=110, minimum width=3cm, minimum height=1cm, text centered, draw=black, fill=blue!30},
  process/.style={rectangle, minimum width=3cm, minimum height=1cm, text centered, draw=black, fill=orange!30},
  decision/.style={diamond, minimum width=3cm, minimum height=1cm, text centered, draw=black, fill=green!30},
  arrow/.style={thick,->,>=stealth},
  file/.style={rectangle, draw=black, fill=white, minimum width=2cm, minimum height=1cm},
  program/.style={rectangle, draw=black, fill=blue!20!white, rounded corners},
  recept/.style={
    draw=green!50!blue!75!black,
    align=left,
    text width=2cm,
  },
  recept_t/.style={
    recept,
    align=center,
    white,
    fill=green!50!blue!75!black,
    draw=green!50!blue!75!black,
  },
  recept_b/.style={
    thick,
    draw=green!50!blue!75!black,
    inner sep=0pt,
  },
  commit/.style={
    draw=green!50!blue!75!black,
    circle,
    minimum width=0.3cm,
  },
}

%\usepackage{minted}
%\usemintedstyle{default} 
%\newcommand{\cpil}[1]{\mintinline{python3}{#1}}

\begin{document}

\frame{\titlepage}

%%%%%%%%%%%%%%%%%%%%%%%%%%%%%%%%%%%%%%%%%%%%%%%%%%%%%%%%%%%%%%%%%%%%%%%%%%%%%%%
%%
%% Постановка проблемы
%%
\begin{frame}{Постановка проблемы}
  \begin{exampleblock}{Последовательный процесс развития проекта}
    \begin{center}
\begin{tikzpicture}[node distance=3.5cm]
\fill[white,draw=gray, dashed] (-1.75cm,1cm) -- (12cm,1cm) -- (12cm,-1.75cm) -- (-1.75cm,-1.75cm) -- cycle;
\node[below right, gray] at (-1.75cm,1cm) {Данные};
\fill[white,draw=gray, dashed] (-1.75cm,-2cm) -- (12cm,-2cm) -- (12cm,-4cm) -- (-1.75cm,-4cm) -- cycle;
\node[above right, gray] at (-1.75cm,-4cm) {Действия};

\node<1->[recept_t] (p1t) at (0.0, 0.2cm) {\textbf{Меню}};
\node<1->[recept, fill=green!20, below=-\pgflinewidth of p1t] (p1milk) {Молоко};
\node<1->[recept_b, fit=(p1t) (p1milk)] {};

\node<3->[recept_t, right of=p1t] (p2t) {\textbf{Меню}};
\node<3->[recept, fill=green!20, below=-\pgflinewidth of p2t] (p2milk) {Молоко};
\node<3->[recept, fill=blue!20, below=-\pgflinewidth of p2milk] (p2soup) {Суп};
\node<3->[recept_b, fit=(p2t) (p2soup)] {};

\node<5->[recept_t, right of=p2t] (p3t) {\textbf{Меню}};
\node<5->[recept, fill=green!20, below=-\pgflinewidth of p3t] (p3milk) {Молоко};
\node<5->[recept, fill=blue!20, below=-\pgflinewidth of p3milk] (p3soup) {Суп};
\node<5->[recept, fill=red!20, below=-\pgflinewidth of p3soup] (p3omelette) {Омлет};
\node<5->[recept_b, fit=(p3t) (p3omelette)] {};

\node<7->[recept_t, right of=p3t] (p4t) {\textbf{Меню}};
\node<7->[recept, fill=yellow!20, below=-\pgflinewidth of p4t] (p4tee) {Чай};
\node<7->[recept, fill=blue!20, below=-\pgflinewidth of p4tee] (p4soup) {Суп};
\node<7->[recept, fill=red!20, below=-\pgflinewidth of p4soup] (p4omelette) {Омлет};
\node<7->[recept_b, fit=(p4t) (p4omelette)] {};

\node<2->[recept, rounded corners, fill=blue!20, below] (p1to2soup) at (1.75cm, -2.5cm) {$+$ Суп};
\node<4->[recept, rounded corners, fill=red!20, below] (p2to3omelette) at (5.25cm, -2.5cm)  {$+$ Омлет};
\node<6->[recept, rectangle split, align=left, rectangle split parts=2, rectangle split part fill={green!20, yellow!20}, draw, text width =2cm, rounded corners, below] (p3to4) at (8.75cm, -2.5cm)  {$-$ Молоко \nodepart{two} $+$ Чай};

\draw<2->[-o] (p1milk) to[out=-90,in=180] (p1to2soup);
\draw<3->[->] (p1to2soup) to[out=0,in=-90] ($(p2soup.south)+(-0.2cm,0)$);
\draw<4->[-o] ($(p2soup.south)+(0.2cm,0)$) to[out=-90,in=180] (p2to3omelette);
\draw<5->[->] (p2to3omelette) to[out=0,in=-90] ($(p3omelette.south)+(-0.2cm,0)$);
\draw<6->[-o] ($(p3omelette.south)+(0.2cm,0)$) to[out=-90,in=180] (p3to4);
\draw<7->[->] (p3to4) to[out=0,in=-90] ($(p4omelette.south)$);

\draw[->, draw, thick] (-2.25cm,-1.875cm) -- (12.5cm,-1.875cm) node[below left] {\textit{время} $t$};

\end{tikzpicture}
    \end{center}
  \end{exampleblock}
\end{frame}

%%%%%%%%%%%%%%%%%%%%%%%%%%%%%%%%%%%%%%%%%%%%%%%%%%%%%%%%%%%%%%%%%%%%%%%%%%%%%%%
%%
%% Постановка проблемы
%%
\begin{frame}{Постановка проблемы}
  \begin{columns}[T]
    \begin{column}[]{0.65\textwidth}  
      \begin{exampleblock}<1->{Последовательный процесс развития проекта}
        \begin{center}
\begin{tikzpicture}[node distance=3.5cm]
\node[recept_t] (p1t) {\textbf{Меню}};
\node[recept, draw=green!50!blue!75!black, fill=green!20, below=-\pgflinewidth of p1t] (p1milk) {Молоко};
\node[recept_b, fit=(p1t) (p1milk)] {};

\node[recept, rounded corners, fill=blue!20, right of=p1t, yshift=0.5cm] (p1to2soup) {$+$ Суп};
\node[recept, rounded corners, fill=red!20, above=0.3cm of p1to2soup] (p2to3omelette) {$+$ Омлет};
\node[recept, rectangle split, align=left, rectangle split parts=2, rectangle split part fill={green!20, yellow!20}, draw, text width =2cm, rounded corners, above=0.3cm of p2to3omelette] (p3to4) {$-$ Молоко \nodepart{two} $+$ Чай};

\node[recept, align=center, white, draw=green!50!blue!75!black, fill=green!50!blue!75!black, right of=p3to4, yshift=0.5cm] (p4t) {\textbf{Меню}};
\node[recept, fill=yellow!20, below=-\pgflinewidth of p4t] (p4tee) {Чай};
\node[recept, fill=blue!20, below=-\pgflinewidth of p4tee] (p4soup) {Суп};
\node[recept, fill=red!20, below=-\pgflinewidth of p4soup] (p4omelette) {Омлет};
\node[recept_b, fit=(p4t) (p4omelette)] {};

\draw[-o] (p1t) to[out=0,in=180] (p1to2soup);
\draw[->] (p1to2soup) -- (p2to3omelette);
\draw[->] (p2to3omelette) -- (p3to4);
\draw[->] (p3to4) to[out=0,in=180] (p4t);
\end{tikzpicture}
        \end{center}
      \end{exampleblock}
    \end{column}
    \begin{column}[]{0.3\textwidth}  
      \begin{block}<1->{Краткая запись}
        \begin{center}
\begin{tikzpicture}[node distance=1cm]
\node[commit, fill=white] (c0) {};
\node[commit, fill=blue!20, above of=c0] (c1) {};
\node[commit, fill=red!20, above of=c1] (c2) {};
\coordinate[above of=c2] (pc2);
\fill[fill=green!20] ($(pc2)+(0.15cm,0)$) arc (0:180:0.15cm) -- cycle;
\fill[fill=yellow!20] ($(pc2)+(0.15cm,0)$) arc (0:-180:0.15cm) -- cycle;
\node[commit] (c3) at (pc2) {};
\draw[->] (c0) -- (c1);
\draw[->] (c1) -- (c2);
\draw[->] (c2) -- (c3);
\end{tikzpicture}
        \end{center}
      \end{block}
    \end{column}
  \end{columns}
\end{frame}

%%%%%%%%%%%%%%%%%%%%%%%%%%%%%%%%%%%%%%%%%%%%%%%%%%%%%%%%%%%%%%%%%%%%%%%%%%%%%%%
%%
%% Постановка проблемы
%%
\begin{frame}{Постановка проблемы}
  \begin{exampleblock}{Параллельный процесс развития проекта}
    \begin{center}
\begin{tikzpicture}[node distance=3.5cm]
\fill[white,draw=gray, dashed] (-1.75cm,0.75cm) -- (12cm,0.75cm) -- (12cm,-4.75cm)
   -- (5.5cm,-4.75cm) -- (5.5cm, -1.25cm) -- (-1.75cm, -1.25cm) -- cycle;
\node[below right, gray] at (-1.75cm,0.75cm) {На работе};

\fill[white,draw=gray, dashed] (-1.75cm,-1.35cm) -- (5.4cm,-1.35cm) -- (5.4cm,-4.75cm) -- (-1.75cm,-4.75cm) -- cycle;
\node[above right, gray] at (-1.75cm,-4.75cm) {Дома};

\node<1->[recept_t] (p1t) {\textbf{Меню}};
\node<1->[recept, fill=green!20, below=-\pgflinewidth of p1t] (p1milk) {Молоко};
\node<1->[recept_b, fit=(p1t) (p1milk)] {};

\node<2->[recept, rounded corners, fill=white, below=1cm of p1milk] (p1copy) {\small Копирование};

\node<2->[recept_t, below=1cm of p1copy] (p21t) {\textbf{Меню}};
\node<2->[recept, fill=green!20, below=-\pgflinewidth of p21t] (p21milk) {Молоко};
\node<2->[recept_b, fit=(p21t) (p21milk)] {};

\node<3->[recept, rounded corners, fill=red!20, right of=p21t] (p2to3omelette) {$+$ Омлет};
\node<3->[recept, rectangle split, align=left, rectangle split parts=2, rectangle split part fill={green!20, yellow!20}, draw, text width =2cm, rounded corners, above=2pt of p2to3omelette] (p3to4) {$-$ Молоко \nodepart{two} $+$ Чай};

\node<4->[recept, rounded corners, fill=blue!20, right of=p1t] (p1to2soup) {$+$ Суп};

\node<5->[recept_t, right of=p1to2soup] (p2t) {\textbf{Меню}};
\node<5->[recept, fill=green!20, below=-\pgflinewidth of p2t] (p2milk) {Молоко};
\node<5->[recept, fill=blue!20, below=-\pgflinewidth of p2milk] (p2soup) {Суп};
\node<5->[recept_b, fit=(p2t) (p2soup)] {};

\node<5->[recept_t, right of=p2to3omelette] (p22t) {\textbf{Меню}};
\node<5->[recept, fill=yellow!20, below=-\pgflinewidth of p22t] (p22tee) {Чай};
\node<5->[recept, fill=red!20, below=-\pgflinewidth of p22tee] (p22omelette) {Омлет};
\node<5->[recept_b, fit=(p22t) (p22omelette)] {};

\node<6->[recept, rounded corners, fill=violet!20] (p3merge) at ($(p2soup.south)!0.5!(p22t.north)$) {\small Слияние};

\node<7->[recept_t, right of=p2t] (p4t) {\textbf{Меню}};
\node<7->[recept, fill=yellow!20, below=-\pgflinewidth of p4t] (p4tee) {Чай};
\node<7->[recept, fill=blue!20, below=-\pgflinewidth of p4tee] (p4soup) {Суп};
\node<7->[recept, fill=red!20, below=-\pgflinewidth of p4soup] (p4omelette) {Омлет};
\node<7->[recept_b, fit=(p4t) (p4omelette)] {};

\node<7->[recept_t, below=1.1cm of p4omelette] (p24t) {\textbf{Меню}};
\node<7->[recept, fill=yellow!20, below=-\pgflinewidth of p24t] (p24tee) {Чай};
\node<7->[recept, fill=red!20, below=-\pgflinewidth of p24tee] (p24omelette) {Омлет};
\node<7->[recept, fill=blue!20, below=-\pgflinewidth of p24omelette] (p24soup) {Суп};
\node<7->[recept_b, fit=(p24t) (p24soup)] {};

\draw<2->[-o] (p1milk) -- (p1copy);
\draw<2->[->] (p1copy) -- (p21t);

\draw<3->[-o] (p21t) -- (p2to3omelette);
\draw<3->[->] (p2to3omelette) to[out=175,in=180] (p3to4);

\draw<4->[-o] (p1t) -- (p1to2soup);

\draw<5->[->] (p1to2soup) -- (p2t);
\draw<5->[->] (p3to4) to[out=0,in=180] (p22t);
\draw<6->[-o] (p2soup) -- (p3merge);
\draw<6->[-o] (p22t) -- (p3merge);

\draw<7->[->] (p3merge) to[out=1,in=180] node[above, yshift=0.35cm]{?} (p4t);
\draw<7->[->] (p3merge) to[out=-1,in=180] node[above]{?} (p24t);

\end{tikzpicture}
    \end{center}
  \end{exampleblock}
\end{frame}

%%%%%%%%%%%%%%%%%%%%%%%%%%%%%%%%%%%%%%%%%%%%%%%%%%%%%%%%%%%%%%%%%%%%%%%%%%%%%%%
%%
%% Постановка проблемы
%%
\begin{frame}{Постановка проблемы}
  \begin{columns}[T]
    \begin{column}[]{0.65\textwidth}  
      \begin{exampleblock}<1->{Параллельный процесс развития проекта}
        \begin{center}
\begin{tikzpicture}[node distance=3.5cm]
\node[recept_t] (p1t) {\textbf{Меню}};
\node[recept, fill=green!20, below=-\pgflinewidth of p1t] (p1milk) {Молоко};
\node[recept_b, fit=(p1t) (p1milk)] {};

\node[recept, rounded corners, fill=blue!20, right=-0.5cm of p1t, yshift=1cm] (p1to2soup) {$+$ Суп};
\node[recept, rounded corners, fill=violet!20, above=2.8cm of p1t] (merge) {Слияние};

\node[recept, rounded corners, fill=red!20, left=-0.5cm of p1t, yshift=1cm] (p2to3omelette) {$+$ Омлет};
\node[recept, rectangle split, align=left, rectangle split parts=2, rectangle split part fill={green!20, yellow!20}, draw, text width =2cm, rounded corners, above=0.3cm of p2to3omelette] (p3to4) {$-$ Молоко \nodepart{two} $+$ Чай};

\node[recept_t, right of=merge, yshift=1.0cm] (p4t) {\textbf{Меню}};
\node[recept, fill=yellow!20, below=-\pgflinewidth of p4t] (p4tee) {Чай};
\node[recept, fill=blue!20, below=-\pgflinewidth of p4tee] (p4soup) {Суп};
\node[recept, fill=red!20, below=-\pgflinewidth of p4soup] (p4omelette) {Омлет};
\node[recept_b, fit=(p4t) (p4omelette)] {};

\draw[-o] (p1t) to[out=0,in=-90] (p1to2soup);
\draw[-o] (p1t) to[out=180,in=-90] (p2to3omelette);
\draw[->] (p2to3omelette) -- (p3to4);
\draw[->] (p3to4) to[out=90,in=185] (merge);
\draw[->] (p1to2soup) to[out=90,in=-5] (merge);
\draw[->] (merge) to[out=90,in=180] (p4t);
\end{tikzpicture}
        \end{center}
      \end{exampleblock}
    \end{column}
    \begin{column}[]{0.3\textwidth}  
      \begin{block}<1->{Краткая запись}
        \begin{center}
\begin{tikzpicture}[node distance=1cm]
\node[commit, fill=white] (c0) {};
\node[commit, fill=blue!20, above of=c0, xshift=1cm] (c1) {};
\node[commit, fill=red!20, above of=c0, xshift=-1cm] (c2) {};
\coordinate[above of=c2] (pc2);
\fill[fill=green!20] ($(pc2)+(0.15cm,0)$) arc (0:180:0.15cm) -- cycle;
\fill[fill=yellow!20] ($(pc2)+(0.15cm,0)$) arc (0:-180:0.15cm) -- cycle;
\node[commit] (c3) at (pc2) {};
\node[commit, fill=violet!20, above of=c3, xshift=1cm] (c4) {};
\draw[->] (c0) -- (c1);
\draw[->] (c0) -- (c2);
\draw[->] (c2) -- (c3);
\draw[->] (c3) -- (c4);
\draw[->] (c1) -- (c1 |- c3) -- (c4);
\end{tikzpicture}
        \end{center}
      \end{block}
    \end{column}
  \end{columns}
\end{frame}

%%%%%%%%%%%%%%%%%%%%%%%%%%%%%%%%%%%%%%%%%%%%%%%%%%%%%%%%%%%%%%%%%%%%%%%%%%%%%%%
%%
%% Определения
%%
\begin{frame}\frametitle{Система управления версиями}
  \begin{block}<1->{Определение}
    \textbf{Система контроля версий} "---  это система, регистрирующая изменения в одном или нескольких файлах с тем, чтобы в дальнейшем была возможность вернуться к определённым старым версиям этих файлов [https://git-scm.com/book/ru/v1/]. 
  \end{block}
  \begin{block}<1->{}
    \textbf{Система управления версиями} "---   программное обеспечение для облегчения работы с изменяющейся информацией. Система управления версиями позволяет хранить несколько версий одного и того же документа, при необходимости возвращаться к более ранним версиям, определять, кто и когда сделал то или иное изменение, и многое другое [wiki]. 
  \end{block}
\end{frame}

%%%%%%%%%%%%%%%%%%%%%%%%%%%%%%%%%%%%%%%%%%%%%%%%%%%%%%%%%%%%%%%%%%%%%%%%%%%%%%%
%%
%% Определения
%%
\begin{frame}\frametitle{Система управления версиями}
  \begin{columns}[T]
    \begin{column}[]{0.45\textwidth}  
      \begin{block}<1->{Основные функции}
        \begin{itemize}
          \item Возможность получения всех версий проекта (файлов).
          \item Хранение полной истории изменений.
          \item Поддержка нескольких версий продукта.
          \item Коллективная работа.
        \end{itemize}
      \end{block}
    \end{column}
    \begin{column}[]{0.45\textwidth}  
      \begin{block}<1->{Дополнительные функции}
        \begin{itemize}
          \item Поддерка нескольких ветвей развития.
          \item Обмен изменениями между ветвями.
          \item Реконфигурация дерева изменений.
          \item Управление правами доступа.
        \end{itemize}
      \end{block}
    \end{column}
  \end{columns}
\end{frame}

%%%%%%%%%%%%%%%%%%%%%%%%%%%%%%%%%%%%%%%%%%%%%%%%%%%%%%%%%%%%%%%%%%%%%%%%%%%%%%%
%%
%% История и обзор
%%
\section{История и обзор систем управления версиями}

%%%%%%%%%%%%%%%%%%%%%%%%%%%%%%%%%%%%%%%%%%%%%%%%%%%%%%%%%%%%%%%%%%%%%%%%%%%%%%%
%%
%% Централизованные системы управления версиями
%%
\begin{frame}{Централизованные системы управления версиями}
  \begin{columns}[T]
    \begin{column}[]{0.45\textwidth}  
      \begin{exampleblock}<1->{Сделанов в}
        \begin{center}
           \Huge Тут было НЛО и забрало изображение
        \end{center}
      \end{exampleblock}
    \end{column}
    \begin{column}[]{0.45\textwidth}  
      \begin{block}<1->{Примеры}
        \begin{itemize}
          \item RCS (Revision Control System) --- система управления пересмотрами версий, локальная, 1985.
          \item CVS (Concurrent Versions System) --- Система управления параллельными версиями, 1994
          \item SVN (Subversion), цель -- заменить собой распространенную на тот момент систему CVS, 2000.
        \end{itemize}
      \end{block}
    \end{column}
  \end{columns}
  \vfill
\end{frame}

%%%%%%%%%%%%%%%%%%%%%%%%%%%%%%%%%%%%%%%%%%%%%%%%%%%%%%%%%%%%%%%%%%%%%%%%%%%%%%%
%%
%% Централизованные системы управления версиями
%%
\begin{frame}{Централизованные системы управления версиями}
  \begin{columns}[T]
    \begin{column}[]{0.45\textwidth}  
      \begin{exampleblock}<1->{Сделанов в}
        \begin{center}
           \Huge Тут было НЛО и забрало изображение
        \end{center}
      \end{exampleblock}
    \end{column}
    \begin{column}[]{0.45\textwidth}  
      \begin{alertblock}<1->{Проблемы}
        \begin{itemize}
          \item История всего проекта развивается последовательно.
        \end{itemize}
      \end{alertblock}
    \end{column}
  \end{columns}
  \vfill
\end{frame}

%%%%%%%%%%%%%%%%%%%%%%%%%%%%%%%%%%%%%%%%%%%%%%%%%%%%%%%%%%%%%%%%%%%%%%%%%%%%%%%
%%
%% Распределённые системы управления версиями
%%
\begin{frame}{Распределённые системы управления версиями}
  \begin{columns}[T]
    \begin{column}[]{0.45\textwidth}  
      \begin{exampleblock}<1->{Сделанов в}
        \begin{center}
           \Huge Тут было НЛО и забрало изображение
        \end{center}
      \end{exampleblock}
    \end{column}
    \begin{column}[]{0.45\textwidth}  
      \begin{block}<1->{Примеры}
        \begin{itemize}
          \item Git
          \item Mercurial
          \item Bazaar
        \end{itemize}
      \end{block}
    \end{column}
  \end{columns}
  \vfill
\end{frame}

%%%%%%%%%%%%%%%%%%%%%%%%%%%%%%%%%%%%%%%%%%%%%%%%%%%%%%%%%%%%%%%%%%%%%%%%%%%%%%%
%%
%% Распределённые системы управления версиями
%%
\begin{frame}{Распределённые системы управления версиями}
  \begin{columns}[T]
    \begin{column}[]{0.45\textwidth}  
      \begin{exampleblock}<1->{Сделанов в}
        \begin{center}
           \Huge Тут было НЛО и забрало изображение
        \end{center}
      \end{exampleblock}
    \end{column}
    \begin{column}[]{0.45\textwidth}  
      \begin{block}<1->{Преимущества}
        \begin{itemize}
          \item ???
        \end{itemize}
      \end{block}
    \end{column}
  \end{columns}
  \vfill
\end{frame}


%%%%%%%%%%%%%%%%%%%%%%%%%%%%%%%%%%%%%%%%%%%%%%%%%%%%%%%%%%%%%%%%%%%%%%%%%%%%%%%
%%
%% 
%%
\begin{frame}{}
  \vfill
  \begin{beamercolorbox}[sep=8pt,center,shadow=true,rounded=true]{title}
    \usebeamerfont{title}Вопросы?
  \end{beamercolorbox}
  \vfill
  \begin{columns}[T]
    \begin{column}[]{0.45\textwidth}  
      \begin{exampleblock}<1->{Сделанов в}
        \begin{center}
           \Huge\LaTeXe
        \end{center}
      \end{exampleblock}
    \end{column}
    \begin{column}[]{0.45\textwidth}  
      \begin{block}<1->{Использовано}
        \begin{itemize}
          \item пакеты: beamer, tikz
          \item строк кода: >350 
        \end{itemize}
      \end{block}
    \end{column}
  \end{columns}
  \vfill
\end{frame}

\end{document}
